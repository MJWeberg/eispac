
\chapter{Level-1 HDF5 File Details}
\label{sec:prep}

This chapter describes in more detail how the EIS level-1 HDF5 files were processed and saved. These HDF5 files can be downloaded from \url{https://eis.nrl.navy.mil/} or by using the search \& download functionality of EISPAC (see \hyperref[sec:download]{Chapter 2}).

\section{Prepping the data in IDL}
The level-0 fits files were prepped using the IDL routine \verb+eis_prep+ available via SolarSoft
\citep{Freeland:1998} with the following options:
\begin{verbatim}
  default = 1
  save = 1
  quiet = 1
  retain = 1
  photons = 1
  refill = 0
\end{verbatim}
There are 400,000+ EIS level-0 files at present, but on a multi-core machine using the IDL bridge
all of the files can be prepped in under 24 hours. We have prepped all of the available EIS files
and saved them to standard fits files in the usual way. Some important points:

\begin{enumerate}
\item[\bf units:] As mentioned previously, the units for the output in these level-1 files is
  ``photon events'' or ``counts.''  This means that the statistical uncertainty can usually be
  estimated as $\sqrt{N}$. When EISPAC loads the HDF5 files, it also calculates an estimate for
  the read noise contribution. It should be noted, however, that the read noise becomes
  significant only at very low flux levels (1--2 counts).

\item[\bf retain:] Note that the retain keyword preserves negative values. One of the jobs of
  \verb+eis_prep+ is to remove the pedestal from the CCD readout and any time-dependent dark
  current. Since the spectral windows are generally narrow, the estimate of the background can be
  too high and the subtracted intensities of the continuum can be negative. This will be dealt with
  during the fitting.

\item[\bf refill:] The warm pixel problem complicates the fitting of EIS line profiles. As
  discussed in the EIS software note \#13 (found in SSW or on the eiswiki), interpolating the values
  of missing pixels appears to best reproduce the original data. This option is left off during
  \verb+eis_prep+ so that the level-1 fits file preserves the information on the missing pixels. As
  discussed below, the interpolation (via the refill option) is done during the read and this data is
  ultimately written to the HDF5 file. A mask indicating which pixels have been interpolated will be
  added to the HDF5 files in a future revision.
\end{enumerate}

Here is an IDL code snippet related to reading the data by looping over the spectral windows.
\begin{lstlisting}[language=idl]
for iwin=0, nwin-1 do begin
  d = eis_getwindata(eis_level1_filename, iwin, /refill, /quiet)
  eis_level1_data[iwin] = ptr_new(d)
endfor
\end{lstlisting}

\section{Writing the HDF5 files}
Each processed level-1 fits file was bundled up with the associated calibration and metadata and saved
as a pair of two HDF5 files:

\begin{enumerate}
\item[\bf eis\_YYYYMMDD\_HHMMSS.data.h5:] Contains only the corrected photon counts within each
  spectral window. This is, by far, the larger of the two HDF5 files. However, they should not need to
  be updated or downloaded very often.
\item[\bf eis\_YYYYMMDD\_HHMMSS.head.h5:] Contains the original fit file index, calibration curves for
  each spectral window (used to convert counts into intensity values), and the corrected pointing
  information.
\end{enumerate}

Users will rarely, if ever, need to access the information inside the HDF5 files directly. EISPAC
contains all of the functions needed to read the data and apply the calibration and pointing
corrections. Nevertheless, the contents and data structure of the HDF5 files are summarized below.

\subsection{.data.h5}
\begin{itemize}
  \item [\bf level1] (group)
  \begin{itemize}
    \item [\bf {intensity\_units}] (dataset) - String with the level1 data units. This will usually be
    "counts"
    \item [\bf {win\#\#}] (dataset) - array of floating point values with the photon counts in a given
    spectral window (e.g. \texttt{win00, win01, ... win24}). Note well, each set of EIS observations may have
    a different number of spectral windows, up to a maximum of 24 windows. Window numbers are
    numbered sequentially from 00; a given wavelength range may be assigned a different window number
    in each EIS study.
  \end{itemize}
\end{itemize}

\subsection{.head.h5}
\begin{itemize}
  \item \textit{Details to be added soon}
\end{itemize}

Additionally, the contents of the HDF5 files can be displayed using
\verb+h5dump+ command line tool, which is provided along with the Anaconda Python distribution
platform or can be installed on its own. Example usage,

\begin{lstlisting}
> h5dump -n eis_20190404_131513.data.h5
FILE_CONTENTS {
group      /
group      /level1
dataset    /level1/intensity_units
dataset    /level1/win00
dataset    /level1/win01
dataset    /level1/win02
dataset    /level1/win03
dataset    /level1/win04
dataset    /level1/win05
. . .
\end{lstlisting}
The actual data associated with each variable can be printed out using the \verb+-d+ option. For
example,
\begin{lstlisting}
> h5dump -d exposure_times/duration eis_20190404_131513.head.h5
HDF5 "eis_20190404_131513.head.h5" {
DATASET "exposure_times/duration" {
   DATATYPE  H5T_IEEE_F32LE
   DATASPACE  SIMPLE { ( 87 ) / ( 87 ) }
   DATA {
   (0): 40.0005, 40.0002, 40.0004, 40.0004, 39.9994, 40.0002, 39.9995, 40,
   (8): 40.0007, 39.9999, 40.0005, 40.0004, 39.9997, 40.0002, 39.9994,
   . . .
}
\end{lstlisting}
