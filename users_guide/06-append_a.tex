
\chapter{Appendix A: mpfit documentation}

\section{parinfo keywords}
\label{sec:parinfo}

\textbf{Constraining Parameter Values with the PARINFO Keyword}

The behavior of MPFIT can be modified with respect to each
parameter to be fitted.  A parameter value can be fixed; simple
boundary constraints can be imposed; limitations on the parameter
changes can be imposed; properties of the automatic derivative can
be modified; and parameters can be tied to one another.

These properties are governed by the PARINFO structure, which is
passed as a keyword parameter to MPFIT.

PARINFO should be a list of dictionaries, one list entry for each parameter.
Each parameter is associated with one element of the array, in
numerical order.  The dictionary can have the following keys
(none are required, keys are case insensitive):

\begin{itemize}

    \item['value' -] the starting parameter value (but see the START\_PARAMS
        parameter for more information).

    \item['fixed' -] a boolean value, whether the parameter is to be held
        fixed or not.  Fixed parameters are not varied by
        MPFIT, but are passed on to MYFUNCT for evaluation.

    \item['limited' -] a two-element boolean array.  If the first/second
        element is set, then the parameter is bounded on the
        lower/upper side.  A parameter can be bounded on both
        sides.  Both LIMITED and LIMITS must be given
        together.

    \item['limits' -] a two-element float array.  Gives the
        parameter limits on the lower and upper sides,
        respectively.  Zero, one or two of these values can be
        set, depending on the values of LIMITED.  Both LIMITED
        and LIMITS must be given together.

    \item['parname' -] a string, giving the name of the parameter.  The
        fitting code of MPFIT does not use this tag in any
        way.  However, the default iterfunct will print the
        parameter name if available.

   \item['step' -] the step size to be used in calculating the numerical
        derivatives.  If set to zero, then the step size is
        computed automatically.  Ignored when AUTODERIVATIVE=0.

    \item['mpside' -] the sidedness of the finite difference when computing
        numerical derivatives.  This field can take four values:
        \begin{itemize}
            \item[0 -] one-sided derivative computed automatically
            \item[1 -] one-sided derivative (f(x+h) - f(x)  )/h
            \item[-1 -] one-sided derivative (f(x)   - f(x-h))/h
            \item[2 -] two-sided derivative (f(x+h) - f(x-h))/(2*h)
        \end{itemize}

        Where "h" is the STEP parameter described above.  The
        "automatic" one-sided derivative method will chose a
        direction for the finite difference which does not
        violate any constraints.  The other methods do not
        perform this check.  The two-sided method is in
        principle more precise, but requires twice as many
        function evaluations.  Default: 0.

    \item['mpmaxstep' -] the maximum change to be made in the parameter
        value.  During the fitting process, the parameter
        will never be changed by more than this value in
        one iteration. A value of 0 indicates no maximum.  Default: 0.

    \item['tied' -] a string expression which "ties" the parameter to other
        free or fixed parameters.  Any expression involving
        constants and the parameter array P are permitted.
        Example: if parameter 2 is always to be twice parameter
        1 then use the following: parinfo(2).tied = '2 * p(1)'.
        Since they are totally constrained, tied parameters are
        considered to be fixed; no errors are computed for them.
        [ NOTE: the PARNAME can't be used in expressions. ]

    \item['mpprint' -] if set to 1, then the default iterfunct will print the
        parameter value.  If set to 0, the parameter value
        will not be printed.  This tag can be used to
        selectively print only a few parameter values out of
        many.  Default: 1 (all parameters printed)
\end{itemize}
